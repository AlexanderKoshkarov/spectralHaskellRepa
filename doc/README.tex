\documentclass[preprint,aip,pop]{article}

\usepackage{graphicx}
\usepackage{amsmath}
\usepackage{amssymb}
\usepackage{float}
\usepackage{color,xcolor}
\usepackage{times}

\date{}

\begin{document}

\title{Galerkin spectral method with Haskell}
\author{Oleksandr Koshkarov\\\small{koshmipt@gmail.com}}

\maketitle

\begin{abstract}
    This is a short description of Galerkin spectral method which I am using to
    solve a system of partial differential equations with Haskell+Repa.
    The purpose of this note is to gather feedback of how to proceed with my code.
\end{abstract}

\section{Notes}
\subsection{What is in this text?}
General notes; the system of equations I am solving; the method I am using to solve them.
\subsection{What is not in this text?}
There is no any Haskell or Physics in this text.
\subsection{What is the purpose of this text?}
I am learning Haskell, and I have encountered a problem --- I am stuck.
All my Haskell experience was about solving toy problems and reading different
books/tutorials. As for now, I feel comfortable solving those toy problems, but
I feel that I cannot make anything significant because I am unfamiliar with
most of Haskell ecosystem. Thus, I decided I should start writing Haskell code no
matter what. 
To do so I picked a research project, which has low time priority for me and
wrote my first Haskell program.
Right now I need a feedback what to do next.
So I decided to write this text about mathematical part of my ``first Haskell
program'' to lure curious and experienced Haskell programmers to look at my code.
\subsection{Main problem}
It seems that I need a more advance time stepper then 4th order Runge Kutta
which I use now. Unfortunately, Haskell does not provide one for Repa to my
knowladge. So my options are
\begin{itemize}
\item Write my own. I am not seriously considering this as writing advanced
  time stepper is very difficult problem.
 \item Use different languages with nice numerical infrastructure. I can
   probably avoid my usual choice C/Fortran by learning something like Julia or
   Ocaml which both have SUNDIALS bindings, but I want to give Haskell one more chance.
 \item Find another Haskell library. Probably that is what I am going to do. The
   most appealing library seems to be petsc-hs by Marco Zocca (ocramz).
\end{itemize}
\subsection{Parallelization}
The code is written with Repa so one can think it makes it parallel.
Unfortunately, the most time consuming part of my code is repa's FFT which is
not parallel, thus if one will use multiple cores, the execution time will increase.
Probably, I need to rewrite it to compute convolution explicitly.
\subsection{About me}
I am a PhD student in the University of Saskatchewan, Canada. My major is
theoretical plasma physics. 
\subsection{Sorry}
This text is a very raw draft, so please be ready for mistakes, typos,
inconsistency and other ugliness.

\section{Problem formulation}
I am trying to solve the system of equations
\begin{subequations}
\begin{align}
& \partial_t n           = -v_0 \partial_x n  + \theta + n \theta + \nabla n \cdot \nabla \chi \\
& \partial_t \theta    = -v_0 \partial_x \theta + \eta - n + 0.5\nabla^2 (\nabla \chi)^2 \\
& \partial_t \eta       = -u_0 \partial_y \theta - \nu (\eta - n) - \rho_s \partial_y \phi + \{ \phi, \eta\} \\
& \mu \nabla^2 \phi = \eta - n \\
&  \nabla^2 \chi = \theta  \\
& \{ \phi , \eta\}  = \partial_x \phi  \partial_y\eta - \partial_y\phi  \partial_x\eta
\end{align}
  \label{mainSystem}
  \end{subequations}
where $n, \theta, \eta, \phi, \chi$ are two $+$ one dimensional functions $(f =
f(t,x,y))$, and $\nu, \mu, v_0, u_0, \rho_s$ are constants. \\
Considered spacial domain is
\begin{align}
  & x  \in [0,L_x]  \\
  & y  \in [0,L_y]
\end{align}
Initial conditions are some fixed functions
\begin{align}
 &  n(0,x,y) = n_0 (x,y) \\
 &  \theta(0,x,y) = \theta_0 (x,y) \\
 &  \eta(0,x,y) = \eta_0 (x,y) \\
 &  \phi(0,x,y) = \phi_0 (x,y) \\
 &  \chi(0,x,y) = \chi_0 (x,y) 
\end{align}
and boundary conditions are periodic
\begin{align}
  & n(t,0,y) = n(t,L_x,y) \\
  & n(t,x,0) = n(t,x,L_y) \\
  & \mbox{Same for } \theta, \eta, \phi, \chi
\end{align}

\section{Spectral method based on Fourier expansion}
To solve out system (\ref{mainSystem}), we approximate our functions
\begin{align}
  & n(t,x,y) \approx \sum_{k_x = -N_{k_x}}^{N_{k_x}} \sum_{k_y = -N_{k_y}}^{N_{k_y}}   \hat{n}_{k_x,k_y}(t) \exp \left ( 2\pi i \left ( \frac{k_x x}{L_x} + \frac{k_y y} {L_y} \right ) \right ) \label{app}
\end{align}
same for $\theta,\eta,\chi,\phi$.
\\ [2mm]
\textbf{Comment} here we used Fourier expansion. One can use
different set of expansion functions, thus using ``different'' type of spectral
method.
\\ [2mm]
Now we can plug (\ref{app}) into (\ref{mainSystem}), then multiply by $\exp
\left (- 2\pi i \left ( \frac{k_x x}{L_x} + \frac{k_y y} {L_y} \right ) \right
)$ and then integrate over the period $\int_0^{L_x} \int_0^{L_y} (***) dxdy$.
This will give us the equation for $\hat{n},\hat{\theta}$\dots
\\ [2mm]
\textbf{Comment} here we used Fourier functions again, to ``project equation'',
but we again could have used different functions.
\\ [2mm]
I could do what I said, with integration. Then nonlinear terms like $n\theta$
would be the convolution of two matrices $\hat{n}$ and $\hat{\theta}$.
The usual way to compute convolution is via fast Fourier transform (FFT), thus I
will make a little bit different derivation which is simpler but a little bit
more ugly. 
\\ [2mm]
\textbf{Comment} if we would derive with integration (for continuous functions),
we would need convolution theorem to transform FFT.
\\ [2mm]
Let say we know our functions only on the mesh
\begin{equation}
  n( m \Delta x, l \Delta y ) = n_{m,l}
\end{equation}
where $m = 0,1,\dots,N_x -1 $ and $l = 0,1,\dots,N_y -1$,
$N_x = 2N_{k_x} + 1$,
$N_y = 2N_{k_y} + 1$,
$\Delta x = L_x/N_x$,
$\Delta y = L_y/N_y$
. Here and later if I state something general for one
function (n) it is true for other functions ($\theta,\eta,\dots$).
I also do not include the last point because it would be the same as first point
due to periodic boundary conditions.
\\
Now we can rewrite equation (\ref{app}) for discrete quantities
\begin{align}
  & n_{m,l} = \mathcal{F}^{-1} \left [ \hat{n}_{k_x,k_y} \right ]_{m,l}  = \frac{1}{N_x N_y}  \sum_{k_x = -N_{k_x}}^{N_{k_x}} \sum_{k_y = -N_{k_y}}^{N_{k_y}}   \hat{n}_{k_x,k_y}(t) \exp \left ( 2\pi i \left ( \frac{k_x m}{N_x} + \frac{k_y l} {N_y} \right ) \right ) 
\end{align}
here I constructed expansion a little bit differently --- I multiplied it by
$1/(N_x N_y)$. I also notice that this expansion is actually two dimensional
reverse FFT. So it has the property
\begin{equation}
  \mathcal{F} \circ \mathcal{F}^{-1} =  \mathcal{F}^{-1} \circ \mathcal{F} =\mathbf{1} \label{rev}
\end{equation}
where $\circ$ is function composition, $\mathbf{1}$ is identity operator and
\begin{align}
  &  \mathcal{F} \left [ n_{m,l} \right ]_{k_x,k_y}  =
    \sum_{m = 0}^{N_{x} -1} \sum_{l = 0}^{N_{y} - 1}
    n_{m,l} \exp \left (- 2\pi i \left ( \frac{k_x m}{N_x} + \frac{k_y l} {N_y} \right ) \right ) 
\end{align}
Ok, so here is the trick, we cannot plug our discrete transformation into our
main system (\ref{mainSystem}), so we plug the transformation (\ref{app}),
differentiate the exponent and only them make it discrete. Then we apply FFT one
more time and use the (\ref{rev}) property.
For linear terms it is simple, we just substitute
\begin{align}
  & \partial_x \rightarrow i K_x \\
  & \partial_y \rightarrow i K_y \\
\end{align}
where $K_{\alpha} = 2\pi k_\alpha/L_\alpha$ (for $\alpha = x,y$).
For nonlinear terms, well, we do nothing.
So we have
\begin{subequations}
\begin{align}
  & \partial_t \hat{n} = -v_0 i K_x  \hat{n}  + \hat{\theta} + NL_1 \\
& \partial_t \hat{\theta}    = -v_0 i K_x  \hat {\theta} + \hat{\eta} - \hat{ n} + NL_2  \\
& \partial_t \hat{\eta}       = -u_0 i K_y \hat{\theta} - \nu (\hat{\eta} - \hat{n}) -
  \rho_s i K_y  \hat{\phi} + NL_3 \\
& -K^2 \mu  \phi = \eta - n \\
&  -K^2 \chi = \theta \\ 
  & K^2 = K_x^2 + K_y^2 \\
  & NL_1 = 
    \mathcal{F} \left [    
\mathcal{F}^{-1} \left [ \hat{n} \right ]  \mathcal{F}^{-1} \left [ \hat{\theta} \right ] 
+
\mathcal{F}^{-1} \left [iK_x \hat{n} \right ]  \mathcal{F}^{-1} \left [ iK_x \hat{\chi} \right ] 
+
\mathcal{F}^{-1} \left [ iK_y \hat{n} \right ]  \mathcal{F}^{-1} \left [ iK_y \hat{\chi} \right ] 
\right ]
\\
  & NL_2 = \mbox{similar to \#1} \\
  & NL_3 = \mbox{similar to \#1} 
\end{align}
\label{mainF}
\end{subequations}
So now I just solve the system (\ref{mainF}) in time with a time stepper algorithm.

\section {Time integration}
For time being I am using Runge-Kutta 4. But I need something better. Currently I
am considering two options: move to different language, use petsc-hs by 
 Marco Zocca (ocramz).

\section{What else?}
The report is not finished.
But hopefully I will improve it, as I go.
So what I have not covered
\begin{itemize}
\item deallising
\item hyperviscosity
\item comparison with theory for linear regime
\end{itemize}


 
\end{document}

